\documentclass{article}
\usepackage[utf8]{inputenc}
\usepackage[spanish]{babel}
\usepackage{listings}
\usepackage{graphicx}
\graphicspath{ {images/} }
\usepackage{cite}

\begin{document}

\begin{titlepage}
    \begin{center}
        \vspace*{1cm}
            
        \Huge
        \textbf{Calistenia}
            
        \vspace{0.5cm}
        \LARGE
        Instrucciones para solucionar un problema
            
        \vspace{1.5cm}
            
        \textbf{Juan Sebastian Anaya Regino}
        
        \vspace{0.9cm}
        \centering
        \includegraphics[width=6cm]{logo.png}
            
        \vfill
            
        \vspace{0.8cm}
            
        \Large
        Despartamento de Ingeniería Electrónica y Telecomunicaciones\\
        Universidad de Antioquia\\
        Medellín\\
        Marzo de 2021
            
    \end{center}
\end{titlepage}

\tableofcontents
\newpage

\section{Desafío}
    \subsection{Preparación}
    Conseguir una hoja de papel no muy grande y dos tarjetas de aproximadamente el tamaño de una cedula de ciudadanía o una tarjeta bancaria.
    Luego de tener haber conseguido las dos tarjetas y la hoja de papel, se deben poner las dos tarjetas encima de una superficie plana y posteriormente se debe colocar la hoja encima de las tarjetas
    
    \subsection{Problema}
    El desafío de la actividad consiste en llevar las dos tarjetas hasta encima de la hoja y formar una especie de pirámide con estas. Las tarjetas deben sostenerse por sí mismas, cabe resaltar que solo se debe usar una sola mano para lograr lo anteriormente dicho.

\section{Solución}
La solución al desafío propuesto se presentara en diez pasos claros y concisos que se describirán a continuación:
    \begin{enumerate}
        \item {Mover la hoja de tal forma que las tarjetas queden a la vista.}
        \item {Mover las tarjetas hasta el borde de la mesa de tal forma que la mitad de las tarjetas quede suspendida en el aire.}
        \item {Sujetar las tarjetas con una mano.}
        \item {Mover la hoja a una posición cómoda.}
        \item {Alinear las tarjetas de modo que parezcan una sola.}
        \item {Apoyar las tarjetas en la hoja.}
        \item {Apoyar una de las tarjetas de forma que actúe como apoyo para la otra.}
        \item {Despacio abrir la tarjeta que no está apoyada de modo que se vaya formando un triángulo.}
        \item {Apoyar la tarjeta que está abriendo en la hoja de modo que pueda mantener el equilibrio.}
        \item {Cuando las dos tarjetas puedan formar un triángulo encima de la hoja deberás soltarlas.}
    \end{enumerate}
Este desafío resalta lo importante que es suministrar instrucciones claras y concisas que se deben ejecutar en un tiempo preciso para que una actividad en concreto se realice correctamente.

\end{document}
